 \documentclass{article}

\usepackage{fontspec}
\usepackage{comment}
\usepackage{amsmath}
% \usepackage{amssymb}
\usepackage{mathtools}
\usepackage{fontspec,xgreek,polyglossia}
\usepackage{graphicx}
\usepackage{float}
\usepackage{hyperref}
\usepackage[pdf]{graphviz}
\usepackage{listings}
\usepackage{minted}
\usepackage{color}
\usepackage{hyperref}
\usepackage{enumitem}
\usepackage{xcolor}

\definecolor{mygreen}{RGB}{28,172,0} % color values Red, Green, Blue
\definecolor{mylilas}{RGB}{170,55,241}

\lstset{language=Matlab,%
    %basicstyle=\color{red},
    breaklines=true,%
    morekeywords={matlab2tikz},
    keywordstyle=\color{blue},%
    morekeywords=[2]{1}, keywordstyle=[2]{\color{black}},
    identifierstyle=\color{black},%
    stringstyle=\color{mylilas},
    commentstyle=\color{mygreen},%
    showstringspaces=false,%without this there will be a symbol in the places where there is a space
    numbers=left,%
    numberstyle={\tiny \color{black}},% size of the numbers
    numbersep=9pt, % this defines how far the numbers are from the text
    emph=[1]{for,end,break},emphstyle=[1]\color{red}, %some words to emphasise
    %emph=[2]{word1,word2}, emphstyle=[2]{style},    
}

\hypersetup{
    colorlinks,
    citecolor=black,
    filecolor=black,
    linkcolor=black,
    urlcolor=black
}
\graphicspath{ {./images/} }
\defaultfontfeatures{Mapping=tex-text}
\setmainfont{Times New Roman}
%\setmainfont{Calibri}
\setdefaultlanguage[variant=modern]{greek}
\setotherlanguage{english}

\title{Άσκηση 2 \\ Ψηφιακές Τηλεπικοινωνίες}
\author{Συμεωνίδης Θεόδωρος 1064870}
\date{Χειμερινό Εξάμηνο, Ακαδημαϊκό έτος 2020-21}
   
\begin{document}

\maketitle

\newpage

\tableofcontents

\newpage

\colorbox{pink}{\textwidth{Ο κώδικας που γράφτηκε έχει επικολληθεί τέλος της εργασίας.}}

\section{Ερώτημα 1}
1. Με βάση τις παραπάνω υποδείξεις, υλοποιήστε το σύστημα Μ-PSK και
αναφερθείτε στα βασικά του σημεία.
\subsection*{Απάντηση στο Ερώτημα 1}
Το M-PSK όπως και η υλοποίηση του δεν διαφέρει ιδιαίτερα από αυτή του M-PAM που υλοποιήσαμε στην προηγούμενη εργασία. Η κύρια διαφορά είναι στο χώρο σημάτων που χρησιμοποιεί για να γίνει η αναπαράσταση του κάθε συμβόλου. Στη περίπτωση του M-PAM ήταν μονοδιάστατος, τώρα είναι δισδιάστατος με βάση τα σήματα $cos(2 \pi f_c t)$ και $-sin(2 \pi f_c t)$. Ο αστερισμός που έχουμε επιλέξει ξεκινάει από το σημείο $s_1 = cos(2 \pi f_c t)$ και τα υπόλοιπα σημεία προκύπτουν ως $s_i = cos(2 \pi f_c t + 2 \pi \dfrac{i-1}{M} )$ δηλαδή είναι M ισαπέχοντα σημεία που βρίσκονται πάνω στο μοναδιαίο κύκλο και άρα έχουν μοναδιαίες ενέργειες $Ε_s$.

Αναφερόμαστε συνοπτικά στα επιμέρους συστατικά της υλοποίησης:
    \begin{itemize}
        \item Mapper \\
        Παίρνει ως είσοδο ένα διάνυσμα από bits μεγέθους $K\times1$, το μετατρέπει σε ένα πίνακα $\dfrac{K}{M}\times M$ και στη συνέχεια σε κάθε γραμμή έχουμε ένα δυαδικό αριθμό τον οποίο τον μετατρέπουμε σε δεκαδικό. Στη περίπτωση της κωδικοποίησης Gray σε αυτό το σημείο μετατρέπουμε τα ακέραια στοιχεία σε κωδικοποιημένα κατά Gray στοιχεία. 
        \item Διαμορφωτή \\
        Κάθε σύμβολο αντιστοιχίζεται στη κατάλληλη κυματομορφή που αντιστοιχεί σε αυτό το σύμβολο $s_m$ του αστερισμού. Σε ένα διάνυσμα όλες αυτές οι τιμές είναι το διάνυσμα εξόδου από τον πομπό.
        \item Κανάλι με AWGN \\
        Στο σημείο αυτό υπολογίζεται ο θόρυβος και προστίθεται στο σήμα εξόδου σύμφωνα με την εκφώνηση.
        \item Αποδιαμορφωτής \\
        Λαμβάνει ως είσοδο το διάνυσμα του σήματος από το κανάλι. Αυτό το σήμα συνήθως είναι μεγαλύτερης διάστασης από τον χώρο σημάτων μας. Οπότε το προβάλει στη βάση σημάτων μας. Αυτό γίνεται υπολογίζοντας το εσωτερικό του γινόμενο με κάθε διάνυσμα βάσης $r_j = <r,\phi_j>$, το οποίο είναι το μέτρο της προβολής του σε κάθε συνιστώσα του χώρου μας ή αλλιώς οι συντεταγμένες του διανύσματος $\overline{r}$ όταν το προβάλουμε στη βάση ${\phi_1, \phi_2, \dots, \phi_j}$. Έτσι ως έξοδο για κάθε σύμβολο έχουμε ένα δισδιάστατο διάνυσμα.
        \item Φωρατής \\
        Το σήμα που αποδιαμορφώθηκε ανήκει στο χώρο σημάτων που έχουμε επιλέξει για τη διαμόρφωση. Ο φωρατής υπολογίζει την Ευκλείδια απόσταση του από κάθε σημείο που ανήκει στον αστερισμό. Αντιστοιχίζει το σημείο μας στο σημείο που έχει την ελάχιστη απόσταση από αυτό, γιατί αυτό μεγιστοποιεί την υπό συνθήκη πιθανότητα $P(S_i|s_j)$, δεδομένου ότι στάλθηκε $s_j$ να αποκωδικοποιήσω σε $s_i$ της Γκαουσιανής στοχαστικής διαδικασίας.
        \item Demapper \\
        Αντιστρέφει τη λειτουργία του Mapper. Στην περίπτωση της κωδικοποίησης κατά Gray εκτελεί την επιπλέον πράξη της αποκωδικοποίησης του διανύσματος των ακεραίων. Τέλος, αντιστοιχίζει κάθε δεκαδικό σύμβολο στα bits στα bits της δυαδική του αναπαράστασης και φτιάχνει ένα διάνυσμα με τα bits των συμβόλων.
    \end{itemize}

\section{Ερώτημα 2} Για καθένα από τα δύο συστήματα, μετρήστε την πιθανότητα σφάλματος και σχεδιάστε τις καμπύλες BER για τιμές του SNR=[0:2:16]dB. Οι καμπύλες BER θα πρέπει να σχεδιαστούν στο ίδιο γράφημα. Στο ίδιο γράφημα, σχεδιάστε και το θεωρητικό BER για το κάθε M-PSK από αυτά που υλοποιήσατε. Σχολιάστε τα αποτελέσματα. Ποιο σύστημα είναι καλύτερο ως προς την πιθανότητα σφάλματος για το ίδιο SNR; Πόσο παραπάνω SNR απαιτείται για να έχει το χειρότερο την ίδια πιθανότητα σφάλματος με το καλύτερο;

\subsection*{Απάντηση στο Ερώτημα 2}

Παρακάτω φαίνονται τις γραφικές παραστάσεις SNR-BER για δυαδική κωδικοποίηση. Έχει υλοποιηθεί και κωδικοποίηση Gray αλλά δεν εισάγουμε τις γραφικές παραστάσεις αυτής καθώς δεν ζητήται. Πρέπει να αναφέρουμε επίσης ότι οι λογαριθμικές γραφικές παραστάσεις έχουν ένα πρόβλημα να γίνουν plot όταν το BER τείνει στο 0 καθώς ο λογάριθμος τείνει στο $-\infty$ και δεν βρήκαμε τρόπο να παραστήσουμε γραφικά την ασύμπτωτη αυτή στο Matlab. \par
Γενικά, παρατηρούμε ότι το 4-PSK συγκλίνει σε μηδενικό BER πιο γρήγορα από το 8-PSK. Αυτό είναι απολύτως φυσιολογικό αφού μειώνεται η απόσταση μεταξύ τον συμβόλων του αστερισμού. Επίσης, παρατηρούμε ότι για SNR=0 έχουν και τα δύο την ίδια (χείριστη) απόδοση, ενώ όσο αυξάνει το SNR τόσο καλύτερα αποδίδει ως προς το BER το 4-PSK, με την μεγαλύτερη διαφορά τους να βρίσκεται όταν έχουν συγκλίνει σε μηδενικό BER και τα δύο συστήματα. Μια ακόμα παρατήρηση είναι ότι η σύγκλιση γίνεται μη γραμμικά, για την ακρίβεια γίνεται λογαριθμικά. Αυτό μπορούμε να το καταλάβουμε γιατί στην περίπτωση που η σύγκλιση γινόταν γραμμικά τότε η γραφική παράσταση θα είχε την μορφή $\log x$, το οποίο δεν ισχύει.\par
Είναι προφανές από τα σχήματα, ότι για το ίδιο SNR το 4-PSK σύστημα είναι καλύτερο από το 8-PSK σύστημα. Ωστόσο όπως γνωρίζουμε το αντίτιμο του μεγαλύτερου SNR που απαιτείται για την εκπομπή του 8-PSK μας προσφέρει μεγαλύτερο ρυθμό μετάδοσης bit, αν κρατήσουμε όλες τις υπόλοιπες παραμέτρους ίδιες. \par
Σχετικά με το πόσο παραπάνω SNR απαιτείται από το 8-PSK (χειρότερο) για να επιτύχει την απόδοση του 4-PSK (καλύτερο) παρατηρούμε ότι όσο μειώνεται το επιθυμητό BER τόσο αυξάνεται και το παραπάνω SNR που απαιτεί το 8-PSK για να επιτύχει την ίδια απόδοση με το 4-PSK. Η μεγαλύτερη διαφορά σε SNR που εμφανίζεται είναι όταν το BER τείνει στο 0 και είναι περίπου 4-4.2dB. 

        \begin{figure}[H]
        \makebox[\textwidth][c]{
        \includegraphics[scale=0.30]{images/ber-mpsk.jpg}}
        \caption{Καμπύλη BER για δυαδική κωδικοποίηση M-PSK, M = [4, 8] bits και SNR = 0:2:16dB}
        \end{figure}
        
\section{Ερώτημα 3} Για καθένα από τα δύο συστήματα, υπολογίστε το φάσμα ισχύος του εκπεμπόμενου σήματος. Σχεδιάστε τα δύο φάσματα στο ίδιο γράφημα σε λογαριθμική κλίμακα. Τι παρατηρείτε ως προς το εύρος ζώνης που απαιτεί το καθένα; Σχολιάστε το εύρος και το πλάτος του κύριου και των δευτερευόντων
λοβών κάθε διαμόρφωσης.

\subsection*{Απάντηση στο Ερώτημα 3}

Στο ερώτημα αυτό αντιμετωπίσαμε πρόβλημα στην εκπόνηση παρά τον χρόνο που αφιερώσαμε στην αποσφαλμάτωση που δεν καρποφόρησε. Ουσιαστικά στην παρακάτω εικόνα φαίνεται το σχήμα της PSD που λαμβάνουμε αν υλοποιήσουμε τον υπολογισμό της σύμφωνα με την εκφώνηση. Το πρόβλημα είναι ότι όταν αλλάζουμε το Μ δεν παρατηρείται διαφοροποίηση στο PSD όπως θα ήταν αναμενόμενο. Έτσι, στο σχήμα παρότι φαίνεται μόνο το 8-PSK αν γίνει αρκετό zoom φαίνεται και το 4-PSK το οποίο ταυτίζεται ακριβώς με το 8-PSK. Αυτό είναι λάθος.

        \begin{figure}[H]
        \makebox[\textwidth][c]{
        \includegraphics[scale=0.50]{images/psd_real.png}}
        \caption{}
        \end{figure}
        
Για την απάντηση του ερωτήματος ως προς το θεωρητικό κομμάτι αποφασίσαμε να πάρουμε κάποιο σχήμα από το διαδίκτυο. Αυτά τα βλέπουμε παρακάτω. Προφανώς το σύστημα μας δεν είναι βασικής ζώνης όπως του σχήματος. Μπορούμε να σκεφτούμε το $2.5 * 10^6$ ως το κέντρο του κεντρικού λοβού. Ουσιαστικά η κυριότερη διαφορά που παρατηρούμε στο 8-PSK σε σχέση με το 4-PSK είναι ότι απαιτούν το ίδιο εύρος ζώνης αλλά έχουν πολύ διαφορετική κατανομή της ισχύος στο εύρος αυτό. Έτσι όσο αυξάνεται το M διασπορά της ισχύος στο φάσμα των συχνοτήτων μειώνεται. Έτσι στην οριακή περίπτωση, θεωρητικά όσο αυξάνεται το M μειώνεται και το εύρος ζώνης που απαιτεί το σύστημα. Επίσης, παρατηρούμε ότι καθώς αυξάνεται το M μειώνεται η ισχύς στον κεντρικό λοβό αλλά και εμφανίζονται περισσότεροι δευτερεύοντες λοβοί. Ουσιαστικά αυτό που συμβαίνει είναι ότι κάθε φορά που το M υψώνεται στο τετράγωνο (π.χ. 2, 4, 16, \dots) τότε στη θέση κάθε λοβού προκύπτουν δύο λοβοί με στο ίδιο ακριβώς πλάτος φάσματος με τον αρχικό αλλά με μικρότερη ισχύ αν τις προσθέσουμε και τις δύο σε σχέση με τον αρχικό. Πιο συγκεκριμένα, στην περίπτωση του 8-PSK ο κεντρικός λοβός του 4-PSK που αποτελείται από 3 υπολοβούς σπάει σε 5 υπολοβούς.

        \begin{figure}[H]
        \makebox[\textwidth][c]{
        \includegraphics[scale=0.50]{images/psd_theoretical.png}}
        \end{figure}
        
        \begin{figure}[H]
        \makebox[\textwidth][c]{
        \includegraphics[scale=0.50]{images/psd_theoretical_1.png}}
        \end{figure}

\section{Ερώτημα 4}
 [Θεωρητική] Παρατηρήστε το Σχήμα 7.57 του βιβλίου (Σχήμα 7.33 στο βιβλίο του Καραγιαννίδη) όπου φαίνονται οι καμπύλες SER για διάφορες τιμές του Μ σε συστήματα M-PSK. Την ίδια μορφή περίπου έχουν και οι αντίστοιχες καμπύλες BER. Τι παρατηρείτε από τα δύο αυτά σχήματα; Σχολιάστε και συγκρίνετε τις διαμορφώσεις M-PSK ως προς το ρυθμό μετάδοσης bits, την πιθανότητα σφάλματος, και το απαιτούμενο εύρος ζώνης όταν αυξάνει το Μ.
 
\subsection*{Απάντηση στο Ερώτημα 4}
Στο βιβλίο του Καραγιαννίδη το σχήμα που λήφθηκε υπόψιν ήταν το Σχήμα 7.33 της σελίδας 526 γιατί το 7.57 δεν ταίριαζε με τα δεδομένα της εκφώνσησης. Επίσης, αφού σύμφωνα με την εκφώνηση θεωρούνται τα BER και SER διαγράμματα ίδια, αν και για παράδειγμα στην περίπτωση που έχουμε Gray κωδικοποίηση με AWGN κανάλι το BER είναι μικρότερο του SER θα αναφερόμαστε μόνο στο SER που είναι και αυτό που απεικονίζεται στο σχήμα του βιβλίου. Αντί για SER μερικές φορές το αναφέρουμε ως πιθανότητα σφάλματος.
\par
Αρχικά, παρατηρούμε ότι για το ίδιο SNR έχουμε μεγαλύτερο SER όσο αυξάνει το M. Αυτό είναι εύκολο να ερμηνευθεί αν λάβει υπόψιν κάποιος ότι κάθε σύμβολο $s_i$ (διάνυσμα στο χώρο σημάτων βάσης) στη πλευρά του δέκτη στο δισδιάστατο χώρο σημάτων μας είναι ένα δείγμα από μια Γκαουσιανή τυχαία μεταβλητή με μέση τιμή τη $s_i$. Δηλαδή το 2δ διάνυσμα είναι δείγμα από τη 2δ Γκαουσιανή τ.μ. με μέση τιμή $s_i$. Υπενθυμίζουμε ότι η φώραση γίνεται λαμβάνοντας υπόψιν την Ευκλείδεια απόσταση του δείγματος από κάθε σύμβολο του αστερισμού. Έτσι το SER είναι ανάλογο της απόστασης των συμβόλων στον αστερισμό. Οπότε για σταθερό λόγο SNR όσο πιο πυκνά τα σύμβολα δηλαδή όσο μεγαλύτερο Μ τόσο μεγαλύτερο SER. Παρατηρούμε ακόμα ότι για μικρότερα Μ το SER ο μηδενισμός της πιθανότητας σφάλματος συμβαίνει με λιγότερο SNR. Αυτό είναι άμεση απόρια του προηγούμενου. Δηλαδή όσο αυξάνεται το SNR τόσο μειώνεται η διασπορά της 2δ τ.μ. και αφού όσο αυξάνεται το M μικραίνει η απόσταση μεταξύ των συμβόλων του αστερισμού είναι λογικό για μικρότερες αποστάσεις (μεγάλο Μ) να απαιτούμε μεγαλύτερα SNR.
\par
%Ρυθμός μετάδωσης bits όταν αυξάνει το Μ
Αναφερόμενοι καθαρά σε ρυθμό μετάδοσης με σταθερές όλες τις άλλες παραμέτρους όπως το SNR. Τότε προφανώς όσο αυξάνει το M υπάρχει μεγαλύτερος ρυθμός μετάδοσης. Αφού ο αστερισμός περιλαμβάνει περισσότερα σύμβολα και άρα μπορούμε να κωδικοποιήσουμε περισσότερα bits σε ένα σύμβολο. Προφανώς ο ρυθμός αποστολής δεν έχει καμία σχέση με το ρυθμό λήψης συμβόλων αφού εκεί πρέπει να λάβουμε υπόψιν και τον παράγοντα SNR. 
\par
%Πιθανότητα σφάλματος όταν αυξάνει το Μ
Για σταθερό SNR καθώς και όλες τις υπόλοιπες παραμέτρους, όσο αυξάνει το M η πιθανότητα σφάλματος αυξάνεται. Αυτό μπορούμε να το διαπιστώσουμε πέρα από το σχήμα, και στο τύπο υπολογισμού του $P_Μ$ που δίνεται στην εκφώνηση της άσκησης.
\par
%Απαιτούμενο εύρος ζώνης όταν αυξάνει το Μ
Παρατηρούμε επίσης ότι όλες οι PSK διαμορφώσεις απαιτούν το ίδιο εύρος ζώνης. Ωστόσο, όσο αυξάνει το M μειώνεται η διασπορά της ισχύος του σήματος στο εύρος ζώνης. Αυτό μπορούμε να το αντιληφθούμε και από τα σχήματα του Ερωτήματος 3. Αυτή η μείωση της διασποράς της ισχύος του σήματος, έχει ως συνέπεια για αρκετά μεγάλα Μ να μπορούμε να επαναχρησιμοποιήσουμε ένα μέρος του εύρους ζώνης για διαμορφώσεις σε γειτονικές φέρουσες συχνότητες.

\section{Κώδικας}
    \lstinputlisting[caption={Πηγαίος κώδικας για \textsf{Erotima\_2.m}}, captionpos=b]
    {code/Erotima_2.m}
    
    \lstinputlisting[caption={Πηγαίος κώδικας για \textsf{Erotima\_3.m}}, captionpos=b]
    {code/Erotima_3.m}
    
    \lstinputlisting[caption={Πηγαίος κώδικας για \textsf{M\_PSK\_Transmitter.m}}, captionpos=b]
    {code/M_PSK_Transmitter.m}
    
    \lstinputlisting[caption={Πηγαίος κώδικας για \textsf{M\_PSK.m}}, captionpos=b]
    {code/M_PSK.m}
\end{document}